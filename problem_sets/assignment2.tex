% Options for packages loaded elsewhere
\PassOptionsToPackage{unicode}{hyperref}
\PassOptionsToPackage{hyphens}{url}
%
\documentclass[
]{article}
\usepackage{amsmath,amssymb}
\usepackage{iftex}
\ifPDFTeX
  \usepackage[T1]{fontenc}
  \usepackage[utf8]{inputenc}
  \usepackage{textcomp} % provide euro and other symbols
\else % if luatex or xetex
  \usepackage{unicode-math} % this also loads fontspec
  \defaultfontfeatures{Scale=MatchLowercase}
  \defaultfontfeatures[\rmfamily]{Ligatures=TeX,Scale=1}
\fi
\usepackage{lmodern}
\ifPDFTeX\else
  % xetex/luatex font selection
\fi
% Use upquote if available, for straight quotes in verbatim environments
\IfFileExists{upquote.sty}{\usepackage{upquote}}{}
\IfFileExists{microtype.sty}{% use microtype if available
  \usepackage[]{microtype}
  \UseMicrotypeSet[protrusion]{basicmath} % disable protrusion for tt fonts
}{}
\makeatletter
\@ifundefined{KOMAClassName}{% if non-KOMA class
  \IfFileExists{parskip.sty}{%
    \usepackage{parskip}
  }{% else
    \setlength{\parindent}{0pt}
    \setlength{\parskip}{6pt plus 2pt minus 1pt}}
}{% if KOMA class
  \KOMAoptions{parskip=half}}
\makeatother
\usepackage{xcolor}
\usepackage[margin=1in]{geometry}
\usepackage{color}
\usepackage{fancyvrb}
\newcommand{\VerbBar}{|}
\newcommand{\VERB}{\Verb[commandchars=\\\{\}]}
\DefineVerbatimEnvironment{Highlighting}{Verbatim}{commandchars=\\\{\}}
% Add ',fontsize=\small' for more characters per line
\usepackage{framed}
\definecolor{shadecolor}{RGB}{248,248,248}
\newenvironment{Shaded}{\begin{snugshade}}{\end{snugshade}}
\newcommand{\AlertTok}[1]{\textcolor[rgb]{0.94,0.16,0.16}{#1}}
\newcommand{\AnnotationTok}[1]{\textcolor[rgb]{0.56,0.35,0.01}{\textbf{\textit{#1}}}}
\newcommand{\AttributeTok}[1]{\textcolor[rgb]{0.13,0.29,0.53}{#1}}
\newcommand{\BaseNTok}[1]{\textcolor[rgb]{0.00,0.00,0.81}{#1}}
\newcommand{\BuiltInTok}[1]{#1}
\newcommand{\CharTok}[1]{\textcolor[rgb]{0.31,0.60,0.02}{#1}}
\newcommand{\CommentTok}[1]{\textcolor[rgb]{0.56,0.35,0.01}{\textit{#1}}}
\newcommand{\CommentVarTok}[1]{\textcolor[rgb]{0.56,0.35,0.01}{\textbf{\textit{#1}}}}
\newcommand{\ConstantTok}[1]{\textcolor[rgb]{0.56,0.35,0.01}{#1}}
\newcommand{\ControlFlowTok}[1]{\textcolor[rgb]{0.13,0.29,0.53}{\textbf{#1}}}
\newcommand{\DataTypeTok}[1]{\textcolor[rgb]{0.13,0.29,0.53}{#1}}
\newcommand{\DecValTok}[1]{\textcolor[rgb]{0.00,0.00,0.81}{#1}}
\newcommand{\DocumentationTok}[1]{\textcolor[rgb]{0.56,0.35,0.01}{\textbf{\textit{#1}}}}
\newcommand{\ErrorTok}[1]{\textcolor[rgb]{0.64,0.00,0.00}{\textbf{#1}}}
\newcommand{\ExtensionTok}[1]{#1}
\newcommand{\FloatTok}[1]{\textcolor[rgb]{0.00,0.00,0.81}{#1}}
\newcommand{\FunctionTok}[1]{\textcolor[rgb]{0.13,0.29,0.53}{\textbf{#1}}}
\newcommand{\ImportTok}[1]{#1}
\newcommand{\InformationTok}[1]{\textcolor[rgb]{0.56,0.35,0.01}{\textbf{\textit{#1}}}}
\newcommand{\KeywordTok}[1]{\textcolor[rgb]{0.13,0.29,0.53}{\textbf{#1}}}
\newcommand{\NormalTok}[1]{#1}
\newcommand{\OperatorTok}[1]{\textcolor[rgb]{0.81,0.36,0.00}{\textbf{#1}}}
\newcommand{\OtherTok}[1]{\textcolor[rgb]{0.56,0.35,0.01}{#1}}
\newcommand{\PreprocessorTok}[1]{\textcolor[rgb]{0.56,0.35,0.01}{\textit{#1}}}
\newcommand{\RegionMarkerTok}[1]{#1}
\newcommand{\SpecialCharTok}[1]{\textcolor[rgb]{0.81,0.36,0.00}{\textbf{#1}}}
\newcommand{\SpecialStringTok}[1]{\textcolor[rgb]{0.31,0.60,0.02}{#1}}
\newcommand{\StringTok}[1]{\textcolor[rgb]{0.31,0.60,0.02}{#1}}
\newcommand{\VariableTok}[1]{\textcolor[rgb]{0.00,0.00,0.00}{#1}}
\newcommand{\VerbatimStringTok}[1]{\textcolor[rgb]{0.31,0.60,0.02}{#1}}
\newcommand{\WarningTok}[1]{\textcolor[rgb]{0.56,0.35,0.01}{\textbf{\textit{#1}}}}
\usepackage{graphicx}
\makeatletter
\def\maxwidth{\ifdim\Gin@nat@width>\linewidth\linewidth\else\Gin@nat@width\fi}
\def\maxheight{\ifdim\Gin@nat@height>\textheight\textheight\else\Gin@nat@height\fi}
\makeatother
% Scale images if necessary, so that they will not overflow the page
% margins by default, and it is still possible to overwrite the defaults
% using explicit options in \includegraphics[width, height, ...]{}
\setkeys{Gin}{width=\maxwidth,height=\maxheight,keepaspectratio}
% Set default figure placement to htbp
\makeatletter
\def\fps@figure{htbp}
\makeatother
\setlength{\emergencystretch}{3em} % prevent overfull lines
\providecommand{\tightlist}{%
  \setlength{\itemsep}{0pt}\setlength{\parskip}{0pt}}
\setcounter{secnumdepth}{-\maxdimen} % remove section numbering
\ifLuaTeX
  \usepackage{selnolig}  % disable illegal ligatures
\fi
\usepackage{bookmark}
\IfFileExists{xurl.sty}{\usepackage{xurl}}{} % add URL line breaks if available
\urlstyle{same}
\hypersetup{
  pdftitle={Assignment 2: 2M1 and 2M2},
  pdfauthor={Madeline Berger},
  hidelinks,
  pdfcreator={LaTeX via pandoc}}

\title{Assignment 2: 2M1 and 2M2}
\author{Madeline Berger}
\date{2026-01-26}

\begin{document}
\maketitle

\subsection{\texorpdfstring{2M1. Recall the globe tossing model from the
chapter. Compute and plot the grid approximate posterior distribution
for \emph{each of the following sets of observations}. In each case,
assume a uniform prior for
p.}{2M1. Recall the globe tossing model from the chapter. Compute and plot the grid approximate posterior distribution for each of the following sets of observations. In each case, assume a uniform prior for p.}}\label{m1.-recall-the-globe-tossing-model-from-the-chapter.-compute-and-plot-the-grid-approximate-posterior-distribution-for-each-of-the-following-sets-of-observations.-in-each-case-assume-a-uniform-prior-for-p.}

\begin{enumerate}
\def\labelenumi{(\arabic{enumi})}
\tightlist
\item
  W, W, W
\item
  W, W, W, L
\item
  L, W, W, L, W, W, W
\end{enumerate}

Steps to grid approximation:

\begin{itemize}
\item
  define the grid, i.e.~how many points to use in estimating the
  posterior, and make a list of the parameter values in the grid
\item
  compute the value of the prior at each parameter value in the grid
\item
  compute the likelihood at each parameter value
\item
  compute the standardized posterior at each parameter value, by
  multiplying the prior by the likelihood
\item
  finally standardize the posterior by dividing each value by the sum of
  all values
\end{itemize}

\textbf{W, W, W}

\begin{Shaded}
\begin{Highlighting}[]
\CommentTok{\# define the grid, uniform prior for p}

\NormalTok{p\_grid }\OtherTok{\textless{}{-}} \FunctionTok{seq}\NormalTok{(}\AttributeTok{from =} \DecValTok{0}\NormalTok{, }\AttributeTok{to =} \DecValTok{1}\NormalTok{, }\AttributeTok{length.out =} \DecValTok{3}\NormalTok{) }\CommentTok{\# start with 3 observations}

\CommentTok{\# define prior }
\NormalTok{prior }\OtherTok{\textless{}{-}} \FunctionTok{rep}\NormalTok{(}\DecValTok{1}\NormalTok{,}\DecValTok{3}\NormalTok{)}

\CommentTok{\#compute likelihood}
\NormalTok{likelihood }\OtherTok{\textless{}{-}} \FunctionTok{dbinom}\NormalTok{(}\DecValTok{3}\NormalTok{, }\AttributeTok{size =} \DecValTok{3}\NormalTok{, }\AttributeTok{prob =}\NormalTok{ p\_grid)}

\CommentTok{\# compute product of likelihood and prior}
\NormalTok{unstd.posterior }\OtherTok{\textless{}{-}}\NormalTok{ likelihood }\SpecialCharTok{*}\NormalTok{ prior}

\CommentTok{\# standardixe Posterior }

\NormalTok{posterior }\OtherTok{\textless{}{-}}\NormalTok{ unstd.posterior }\SpecialCharTok{/} \FunctionTok{sum}\NormalTok{(unstd.posterior)}


\FunctionTok{plot}\NormalTok{( p\_grid , posterior , }\AttributeTok{type=}\StringTok{"b"}\NormalTok{ ,}
\AttributeTok{xlab=}\StringTok{"probability of water"}\NormalTok{ , }\AttributeTok{ylab=}\StringTok{"posterior probability"}\NormalTok{ )}
\FunctionTok{mtext}\NormalTok{( }\StringTok{"3 points {-} WWW"}\NormalTok{ )}
\end{Highlighting}
\end{Shaded}

\includegraphics{assignment2_files/figure-latex/unnamed-chunk-1-1.pdf}

\textbf{W, W, W, L}

\begin{Shaded}
\begin{Highlighting}[]
\CommentTok{\# define the grid, uniform prior for p}

\NormalTok{p\_grid }\OtherTok{\textless{}{-}} \FunctionTok{seq}\NormalTok{(}\AttributeTok{from =} \DecValTok{0}\NormalTok{, }\AttributeTok{to =} \DecValTok{1}\NormalTok{, }\AttributeTok{length.out =} \DecValTok{4}\NormalTok{) }\CommentTok{\# 4 observations}

\CommentTok{\# define prior }
\NormalTok{prior }\OtherTok{\textless{}{-}} \FunctionTok{c}\NormalTok{(}\FunctionTok{rep}\NormalTok{(}\DecValTok{1}\NormalTok{,}\DecValTok{3}\NormalTok{),}\DecValTok{0}\NormalTok{)}

\CommentTok{\#compute likelihood}
\NormalTok{likelihood }\OtherTok{\textless{}{-}} \FunctionTok{dbinom}\NormalTok{(}\DecValTok{3}\NormalTok{, }\AttributeTok{size =} \DecValTok{4}\NormalTok{, }\AttributeTok{prob =}\NormalTok{ p\_grid)}

\CommentTok{\# compute product of likelihood and prior}
\NormalTok{unstd.posterior }\OtherTok{\textless{}{-}}\NormalTok{ likelihood }\SpecialCharTok{*}\NormalTok{ prior}

\CommentTok{\# standardixe Posterior }

\NormalTok{posterior }\OtherTok{\textless{}{-}}\NormalTok{ unstd.posterior }\SpecialCharTok{/} \FunctionTok{sum}\NormalTok{(unstd.posterior)}


\FunctionTok{plot}\NormalTok{( p\_grid , posterior , }\AttributeTok{type=}\StringTok{"b"}\NormalTok{ ,}
\AttributeTok{xlab=}\StringTok{"probability of water"}\NormalTok{ , }\AttributeTok{ylab=}\StringTok{"posterior probability"}\NormalTok{ )}
\FunctionTok{mtext}\NormalTok{( }\StringTok{"4 points {-} W W W L"}\NormalTok{ )}
\end{Highlighting}
\end{Shaded}

\includegraphics{assignment2_files/figure-latex/unnamed-chunk-2-1.pdf}

\textbf{L, W, W, L, W, W, W}

\begin{Shaded}
\begin{Highlighting}[]
\CommentTok{\# define the grid, uniform prior for p}

\NormalTok{p\_grid }\OtherTok{\textless{}{-}} \FunctionTok{seq}\NormalTok{(}\AttributeTok{from =} \DecValTok{0}\NormalTok{, }\AttributeTok{to =} \DecValTok{1}\NormalTok{, }\AttributeTok{length.out =} \DecValTok{7}\NormalTok{) }\CommentTok{\# 7 observations}

\CommentTok{\# define prior {-} not sure I did this right}
\NormalTok{prior }\OtherTok{\textless{}{-}} \FunctionTok{c}\NormalTok{(}\DecValTok{0}\NormalTok{,}\FunctionTok{rep}\NormalTok{(}\DecValTok{1}\NormalTok{,}\DecValTok{2}\NormalTok{),}\DecValTok{0}\NormalTok{,}\FunctionTok{rep}\NormalTok{(}\DecValTok{1}\NormalTok{,}\DecValTok{3}\NormalTok{))}

\CommentTok{\#compute likelihood}
\NormalTok{likelihood }\OtherTok{\textless{}{-}} \FunctionTok{dbinom}\NormalTok{(}\DecValTok{4}\NormalTok{, }\AttributeTok{size =} \DecValTok{7}\NormalTok{, }\AttributeTok{prob =}\NormalTok{ p\_grid)}

\CommentTok{\# compute product of likelihood and prior}
\NormalTok{unstd.posterior }\OtherTok{\textless{}{-}}\NormalTok{ likelihood }\SpecialCharTok{*}\NormalTok{ prior}

\CommentTok{\# standardixe Posterior }

\NormalTok{posterior }\OtherTok{\textless{}{-}}\NormalTok{ unstd.posterior }\SpecialCharTok{/} \FunctionTok{sum}\NormalTok{(unstd.posterior)}


\FunctionTok{plot}\NormalTok{( p\_grid , posterior , }\AttributeTok{type=}\StringTok{"b"}\NormalTok{ ,}
\AttributeTok{xlab=}\StringTok{"probability of water"}\NormalTok{ , }\AttributeTok{ylab=}\StringTok{"posterior probability"}\NormalTok{ )}
\FunctionTok{mtext}\NormalTok{( }\StringTok{"7 points {-} L, W, W, L, W, W, W"}\NormalTok{ )}
\end{Highlighting}
\end{Shaded}

\includegraphics{assignment2_files/figure-latex/unnamed-chunk-3-1.pdf}

\subsection{2M2. Now assume a prior for p that is equal to zero when p
\textless{} 0.5 and is a positive constant when p ≥0.5. Again compute
and plot the grid approximate posterior distribution for each of the
sets of observations in the problem just
above.}\label{m2.-now-assume-a-prior-for-p-that-is-equal-to-zero-when-p-0.5-and-is-a-positive-constant-when-p-0.5.-again-compute-and-plot-the-grid-approximate-posterior-distribution-for-each-of-the-sets-of-observations-in-the-problem-just-above.}

\textbf{W, W, W}

\begin{Shaded}
\begin{Highlighting}[]
\CommentTok{\# define the grid, uniform prior for p}

\NormalTok{p\_grid }\OtherTok{\textless{}{-}} \FunctionTok{seq}\NormalTok{(}\AttributeTok{from =} \DecValTok{0}\NormalTok{, }\AttributeTok{to =} \DecValTok{1}\NormalTok{, }\AttributeTok{length.out =} \DecValTok{3}\NormalTok{) }

\CommentTok{\# define new prior }
\NormalTok{prior }\OtherTok{\textless{}{-}} \FunctionTok{ifelse}\NormalTok{(p\_grid }\SpecialCharTok{\textless{}} \FloatTok{0.5}\NormalTok{, }\DecValTok{0}\NormalTok{, }\DecValTok{1}\NormalTok{)}

\CommentTok{\#compute likelihood}
\NormalTok{likelihood }\OtherTok{\textless{}{-}} \FunctionTok{dbinom}\NormalTok{(}\DecValTok{3}\NormalTok{, }\AttributeTok{size =} \DecValTok{3}\NormalTok{, }\AttributeTok{prob =}\NormalTok{ p\_grid)}

\CommentTok{\# compute product of likelihood and prior}
\NormalTok{unstd.posterior }\OtherTok{\textless{}{-}}\NormalTok{ likelihood }\SpecialCharTok{*}\NormalTok{ prior}

\CommentTok{\# standardixe Posterior }

\NormalTok{posterior }\OtherTok{\textless{}{-}}\NormalTok{ unstd.posterior }\SpecialCharTok{/} \FunctionTok{sum}\NormalTok{(unstd.posterior)}


\FunctionTok{plot}\NormalTok{( p\_grid , posterior , }\AttributeTok{type=}\StringTok{"b"}\NormalTok{ ,}
\AttributeTok{xlab=}\StringTok{"probability of water"}\NormalTok{ , }\AttributeTok{ylab=}\StringTok{"posterior probability"}\NormalTok{ )}
\FunctionTok{mtext}\NormalTok{( }\StringTok{"3 points {-} WWW {-} new Prior "}\NormalTok{ )}
\end{Highlighting}
\end{Shaded}

\includegraphics{assignment2_files/figure-latex/unnamed-chunk-4-1.pdf}

\textbf{W, W, W, L}

\begin{Shaded}
\begin{Highlighting}[]
\CommentTok{\# define the grid, uniform prior for p}

\NormalTok{p\_grid }\OtherTok{\textless{}{-}} \FunctionTok{seq}\NormalTok{(}\AttributeTok{from =} \DecValTok{0}\NormalTok{, }\AttributeTok{to =} \DecValTok{1}\NormalTok{, }\AttributeTok{length.out =} \DecValTok{4}\NormalTok{) }\CommentTok{\# 4 observations}

\CommentTok{\# define prior }
\NormalTok{prior }\OtherTok{\textless{}{-}} \FunctionTok{ifelse}\NormalTok{(p\_grid }\SpecialCharTok{\textless{}} \FloatTok{0.5}\NormalTok{, }\DecValTok{0}\NormalTok{, }\DecValTok{1}\NormalTok{)}

\CommentTok{\#compute likelihood}
\NormalTok{likelihood }\OtherTok{\textless{}{-}} \FunctionTok{dbinom}\NormalTok{(}\DecValTok{3}\NormalTok{, }\AttributeTok{size =} \DecValTok{4}\NormalTok{, }\AttributeTok{prob =}\NormalTok{ p\_grid)}

\CommentTok{\# compute product of likelihood and prior}
\NormalTok{unstd.posterior }\OtherTok{\textless{}{-}}\NormalTok{ likelihood }\SpecialCharTok{*}\NormalTok{ prior}

\CommentTok{\# standardixe Posterior }

\NormalTok{posterior }\OtherTok{\textless{}{-}}\NormalTok{ unstd.posterior }\SpecialCharTok{/} \FunctionTok{sum}\NormalTok{(unstd.posterior)}


\FunctionTok{plot}\NormalTok{( p\_grid , posterior , }\AttributeTok{type=}\StringTok{"b"}\NormalTok{ ,}
\AttributeTok{xlab=}\StringTok{"probability of water"}\NormalTok{ , }\AttributeTok{ylab=}\StringTok{"posterior probability"}\NormalTok{ )}
\FunctionTok{mtext}\NormalTok{( }\StringTok{"4 points {-} W W W L {-} New prior"}\NormalTok{ )}
\end{Highlighting}
\end{Shaded}

\includegraphics{assignment2_files/figure-latex/unnamed-chunk-5-1.pdf}

\textbf{L, W, W, L, W, W, W}

\begin{Shaded}
\begin{Highlighting}[]
\CommentTok{\# define the grid, uniform prior for p}

\NormalTok{p\_grid }\OtherTok{\textless{}{-}} \FunctionTok{seq}\NormalTok{(}\AttributeTok{from =} \DecValTok{0}\NormalTok{, }\AttributeTok{to =} \DecValTok{1}\NormalTok{, }\AttributeTok{length.out =} \DecValTok{7}\NormalTok{) }\CommentTok{\# 7 observations}

\CommentTok{\# define prior {-} not sure I did this right}
\NormalTok{prior }\OtherTok{\textless{}{-}} \FunctionTok{ifelse}\NormalTok{(p\_grid }\SpecialCharTok{\textless{}} \FloatTok{0.5}\NormalTok{, }\DecValTok{0}\NormalTok{, }\DecValTok{1}\NormalTok{)}

\CommentTok{\#compute likelihood}
\NormalTok{likelihood }\OtherTok{\textless{}{-}} \FunctionTok{dbinom}\NormalTok{(}\DecValTok{4}\NormalTok{, }\AttributeTok{size =} \DecValTok{7}\NormalTok{, }\AttributeTok{prob =}\NormalTok{ p\_grid)}

\CommentTok{\# compute product of likelihood and prior}
\NormalTok{unstd.posterior }\OtherTok{\textless{}{-}}\NormalTok{ likelihood }\SpecialCharTok{*}\NormalTok{ prior}

\CommentTok{\# standardixe Posterior }

\NormalTok{posterior }\OtherTok{\textless{}{-}}\NormalTok{ unstd.posterior }\SpecialCharTok{/} \FunctionTok{sum}\NormalTok{(unstd.posterior)}


\FunctionTok{plot}\NormalTok{( p\_grid , posterior , }\AttributeTok{type=}\StringTok{"b"}\NormalTok{ ,}
\AttributeTok{xlab=}\StringTok{"probability of water"}\NormalTok{ , }\AttributeTok{ylab=}\StringTok{"posterior probability"}\NormalTok{ )}
\FunctionTok{mtext}\NormalTok{( }\StringTok{"7 points {-} L, W, W, L, W, W, W {-} New prior"}\NormalTok{ )}
\end{Highlighting}
\end{Shaded}

\includegraphics{assignment2_files/figure-latex/unnamed-chunk-6-1.pdf}

\end{document}
